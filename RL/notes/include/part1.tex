\part{Multi-arm bandits}
K-摇臂赌博机在选择一个动作后,有一定的概率获得奖励,但是这个概率是我们不知道的,设在时间$t$采取的动作为$A_t$,
获得的奖励为$R_t$,使用$p_* (a)$表示动作$a$预计的奖励则有$$p_* (a)=\boldsymbol{E}[R_t|A_t =a]$$
假设我们得到动作$a$的奖励的估计,我们每次采取奖励最大的动作。我们记在时间$t$,动作$a$的奖励为
$Q_t(a) \approx q_* (a)$, 每次采取奖励最大的动作作为当前动作,即使用贪心算法。在试探的过程,有两种策略,
一种称之为探索,每个动作均有机会被试探,能够获得每个动作的奖励的估计,一种为利用,即,每次采取奖励最高的动作,
能够较为迅速的找到最优的动作策略。

\section{Action-Value Method}
我们需要获得每一个动作的奖励的估计,一种简单的方法就是使用平均数:
$$ Q_t(a)=\frac{ \sum_{i=1}^{t-1} R_i \cdot \boldsymbol{1}_{A_i=a} }{ \sum_{i=1}^{t-1}\boldsymbol{1}_{A_i=a} } $$
为了平衡探索和利用,我们在贪心算法中以概率$\epsilon$随机选择动作,最终的决策为:
$$
A_t=
\begin{cases}
  \mathop{\arg\max}_a Q_t(a) & \mbox { possibility is 1 - \epsilon }\\
  random & \mbox{possibility is \epsilon}
\end{cases}
$$
算法步骤:
\begin{algorithm}[H]
    \caption{$\epsilon-greedy$}
    \KwIn{摇臂数 K;奖赏函数 R;尝试次数 T;探索概率 \epsilon}
    \KwOut{累计奖励 r}

    r=0\;
    $\forall a =1,2,\cdots,K:Q_0(a)=0,count(a)=0$\;

\end{algorithm}
